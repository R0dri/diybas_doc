

\chapter{Marco Referencial}
\section{Introducci\'on}

En la introducci\'{o}n, el autor presenta y se\~{n}ala la importancia, el origen (los antecedentes te\'{o}ricos y pr\'{a}cticos), los objetivos, los alcances, las limitaciones, la metodolog\'{\i}a empleada, el significado que el estudio tiene en el avance del campo respectivo y su aplicaci\'{o}n en el \'{a}rea investigada. No debe confundirse con el resumen y se recomienda que la introducci\'{o}n tenga una extensi\'{o}n de m\'{\i}nimo 2 p\'{a}ginas y m\'{a}ximo de 4 p\'{a}ginas.\\

La presente plantilla maneja una familia de fuentes utilizada generalmente en LaTeX, conocida como Computer Modern, espec\'{\i}ficamente LMRomanM para el texto de los p\'{a}rrafos y CMU Sans Serif para los t\'{\i}tulos y subt\'{\i}tulos. Sin embargo, es posible sugerir otras fuentes tales como Garomond, Calibri, Cambria, Arial o Times New Roman, que por claridad y forma, son adecuadas para la edici\'{o}n de textos acad\'{e}micos.\\

La presente plantilla tiene en cuenta aspectos importantes de la Norma T\'{e}cnica Colombiana - NTC 1486, con el fin que sea usada para la presentaci\'{o}n final de las tesis de maestr\'{\i}a y doctorado y especializaciones y especialidades en el \'{a}rea de la salud, desarrolladas en la Universidad Nacional de Colombia.\\

Las m\'{a}rgenes, numeraci\'{o}n, tama\~{n}o de las fuentes y dem\'{a}s aspectos de formato, deben ser conservada de acuerdo con esta plantilla, la cual esta dise\~{n}ada para imprimir por lado y lado en hojas tama\~{n}o carta. Se sugiere que los encabezados cambien seg\'{u}n la secci\'{o}n del documento (para lo cual esta plantilla esta construida por secciones).\\

Si se requiere ampliar la informaci\'{o}n sobre normas adicionales para la escritura se puede consultar la norma NTC 1486 en la Base de datos del ICONTEC (Normas T\'{e}cnicas Colombianas) disponible en el portal del SINAB de la Universidad Nacional de Colombia\footnote{ver: www.sinab.unal.edu.co}, en la secci\'{o}n "Recursos bibliogr\'{a}ficos" opci\'{o}n "Bases de datos".  Este portal tambi\'{e}n brinda la posibilidad de acceder a un instructivo para la utilizaci\'{o}n de Microsoft Word y Acrobat Professional, el cual est\'{a} disponible en la secci\'{o}n "Servicios", opci\'{o}n "Tr\'{a}mites" y enlace "Entrega de tesis".\\

La redacci\'{o}n debe ser impersonal y gen\'{e}rica. La numeraci\'{o}n de las hojas sugiere que las p\'{a}ginas preliminares se realicen en n\'{u}meros romanos en may\'{u}scula y las dem\'{a}s en n\'{u}meros ar\'{a}bigos, en forma consecutiva a partir de la introducci\'{o}n que comenzar\'{a} con el n\'{u}mero 1. La cubierta y la portada no se numeran pero si se cuentan como p\'{a}ginas.\\

Para trabajos muy extensos se recomienda publicar m\'{a}s de un volumen. Se debe tener en cuenta que algunas facultades tienen reglamentada la extensi\'{o}n m\'{a}xima de las tesis  o trabajo de investigaci\'{o}n; en caso que no sea as\'{\i}, se sugiere que el documento no supere 120 p\'{a}ginas.\\

No se debe utilizar numeraci\'{o}n compuesta como 13A, 14B \'{o} 17 bis, entre otros, que indican superposici\'{o}n de texto en el documento. Para resaltar, puede usarse letra cursiva o negrilla. Los t\'{e}rminos de otras lenguas que aparezcan dentro del texto se escriben en cursiva.\\

\section{Planteamiento del Problema}
\subsection{Definici\'on del problema}

\section{Objetivos}
\subsection{Objetivo General}
\subsection{Objetivos espec\'ificos}
\begin{itemize}
	
	\item Objetivo espec\'ifico 1 
	\item Objetivo espec\'ifico 2 
	\item Objetivo espec\'ifico 3 
	\item Objetivo espec\'ifico 4 
	\item Objetivo espec\'ifico 5 
	\item Objetivo espec\'ifico 6 
\end{itemize}


\section{Justificaci\'on}
\section{L\'imites y Alcances}
\subsection{L\'imites}
\begin{itemize}
	
	\item L\'imite 1 
	\item L\'imite 2 
	\item L\'imite 3 
	\item L\'imite 4 
	\item L\'imite 5 
	\item L\'imiteo 6 
\end{itemize}
\subsection{Alcances}

\begin{itemize}
	
	\item Alcance 1 
	\item Alcance 2
	\item Alcance 3
	\item Alcance 4
	\item Alcance 5
	\item Alcance 6
\end{itemize}
